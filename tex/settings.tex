\usepackage[a4paper, 
            left=2.0cm, 
            right=1.7cm, 
            top=3cm, 
            bottom=2.0cm,
            marginparwidth=1cm]{geometry}
% following two lines suggested by Joh Kitchin 
% for writing nice documents with well-spaced lines
\usepackage{setspace}
\onehalfspacing

\setlength{\parindent}{20pt}
\usepackage[utf8]{inputenc}
\usepackage{amsthm,amsmath,amssymb,amsfonts}
\usepackage{csquotes}
\usepackage{titling}
\setlength{\droptitle}{-60pt}
\usepackage{marginnote}
% to take away the error unknown float option `H'
\usepackage{float}             
% Making caption font smaller on figures and tables.  
% https://stackoverflow.com/a/27243065/505306
\usepackage{caption} 
\captionsetup{font=footnotesize}
\usepackage[framemethod=TikZ]{mdframed}
\usepackage{tocloft}
% Adding a line at the top of the table of contents saying page
\addtocontents{toc}{~\hfill\textbf{Page}\par}
\renewcommand\cftsecleader{\cftdotfill{\cftdotsep}}
\usepackage[titletoc,toc,page]{appendix}
% The great microtype package for better word adjustment 
%https://tex.stackexchange.com/a/586  
\usepackage[stretch=10]{microtype} 
% For aligning itemize environments to left 
% https://tex.stackexchange.com/a/278199/17858
\usepackage{enumitem}               
% if you want to create a new list from scratch  
% https://tex.stackexchange.com/q/2291/17858
\newlist{alphalist}{enumerate}{1}      
% in that case, at least label must be specified using \setlist
\setlist[alphalist,1]{label=\textbf{\Alph*.}} 
\usepackage[english]{babel}
\setlength{\headsep}{5pt}
\usepackage{blindtext,kantlipsum}
%\usepackage{IEEEtrantools}
\usepackage{ieeetrantools}
\usepackage{mathtools}
\usepackage{listings}
\usepackage[inline]{asymptote}
\usepackage{asypictureB}
\usepackage{filecontents}
\usepackage{parskip} 
\usepackage{tikz,graphicx}                            
% For lists in two or more columns
\usepackage{multicol}  
\usepackage[nokwfunc, ruled]{algorithm2e}
\newcommand\mycommfont[1]{\footnotesize\ttfamily\textcolor{blue}{#1}}
\usepackage[T1]{fontenc}
% So that sections/code-blocks don't straddle two pages 
\usepackage{needspace}  
\usepackage{mathtools}
\usepackage{subfig}
\usepackage{etoolbox} 
\usepackage{color}
\usepackage{pifont}
% For italicizing quotes: https://tex.stackexchange.com/a/288556/17858, 
% from https://tex.stackexchange.com/a/391739
\usepackage{quoting,xparse} 
\NewDocumentCommand{\bywhom}{m}{% the Bourbaki trick
  {\nobreak\hfill\penalty50\hskip1em\null\nobreak
   \hfill\mbox{\normalfont(#1)}%
   \parfillskip=0pt \finalhyphendemerits=0 \par}%
}
\NewDocumentEnvironment{pquotation}{m}
  {\begin{quoting}[
     indentfirst=true,
     leftmargin=\parindent,
     rightmargin=\parindent]\itshape}
  {\bywhom{#1}\end{quoting}}
% Convenience commands
\newcommand*\circled[1]{\tikz[baseline=(char.base)]{\node[shape=circle,draw,inner sep=2pt] (char) {#1};}}
\providecommand{\myceil}[1]{\left \lceil #1 \right \rceil }	    % Ceil function
% Floor function\renewcommand{\labelitemi}{\tiny$\blacksquare$}	
\providecommand{\myfloor}[1]{\left \lfloor #1 \right \rfloor }
% for drawing the conditional probability `|` sign neatly.
\newcommand\given[1][]{\:#1\vert\:} 
\newcommand\RR{\mathbb{R}}				
\newcommand\CC{\mathbb{C}}			
\newcommand\ZZ{\mathbb{Z}}		
\newcommand\NN{\mathbb{N}}	
\newcommand\rarr{\rightarrow}
\newcommand\larr{\leftarrow}	
\newcommand\defeq{\coloneqq}% := symbol
\renewcommand\tilde{ \: \thicksim \: }% Sane tildas
\newmdenv[topline=false, 
	  bottomline=false, 
	  skipabove=\topsep,
	  skipbelow=\topsep]{siderules}
% https://tex.stackexchange.com/a/458876/17858 
% rounded pink rectangles around an inline word
\newcommand{\sticker}[1]{\tikz[baseline=(X.base)]
	                 \node [draw=red,
			 	fill=pink!60,
				semithick,
                                rectangle,
				inner sep=2pt, 
				rounded corners=3pt](X){{\footnotesize \color{red} #1}};} 
\newif\ifshowcode
\showcodetrue
\usepackage{latexsym}
\usepackage{listings}
\usepackage{color}
\definecolor{linkcolor}{rgb}{0.1, 0.60, 0.20}
\usepackage{todonotes}
\usepackage{booktabs}
\usepackage{longtable}
\usepackage[%
      raiselinks,%
      pdfhighlight=/O,%
      hyperfigures,%
      breaklinks,%
      colorlinks,%
      pdfstartview=FitBH,%
      linkcolor={linkcolor},%
      anchorcolor={linkcolor},%
      citecolor={linkcolor},%
      filecolor={linkcolor},%
      menucolor={linkcolor},%
      urlcolor={linkcolor}%
   ]{hyperref}
% Taken from https://tex.stackexchange.com/a/371469 
% for drawing a nice rule across the page
\newcommand\myrule{\par\noindent\rule{\textwidth}{0.4pt}}
\usepackage{xcolor}
\usepackage{sectsty}
%\allsectionsfont{\sffamily}
\definecolor{lava}{rgb}{0.81, 0.06, 0.13}
\definecolor{mahogany}{rgb}{0.75, 0.25, 0.0}
\definecolor{sacramentostategreen}{rgb}{0.0, 0.34, 0.25}
\setcounter{tocdepth}{2}
\usepackage{marginnote}
\newcommand\margin[1]{\marginnote{\color{cadmiumgreen}{#1} }}
\definecolor{cadmiumgreen}{rgb}{0.0, 0.42, 0.24}
% Squaremarkers for bullets of itemized lists
\usepackage{wasysym}
\usepackage{marvosym}
%\renewcommand{\labelitemi}{\scriptsize$\blacksquare$}
\renewcommand{\labelitemi}{\ding{118}} % 4 squares in the form of a diamond
% To express an idea in a crunchy way.
\newcommand{\crunchy}[1]{\lbrack{} \large \textit{#1} \normalsize \rbrack}
% For commenting out block of text, that you still 
% want syntax highlighted in the LaTeX file
\newcommand{\remove}[1]{} 
% For page numbers at top right. Found this here
% https://tex.stackexchange.com/a/56321
\pagestyle{myheadings}
% For left aligning title and authors
% https://tex.stackexchange.com/q/85343
\makeatletter
\usepackage[us,12hr]{datetime}
\renewcommand{\maketitle}{\bgroup\setlength{\parindent}{0pt}
\begin{flushleft}
  \textbf{\LARGE \@title} \\ \vspace{2mm} 
  \large \@author  \\ \vspace{3.5mm} 
  \large \@date{} \\
  \currenttime \normalsize
\end{flushleft}\egroup
}
\makeatother
% https://tex.stackexchange.com/a/192504 
% Force itemize inside description onto a new line
\setlist[itemize]{topsep=0pt,before=\leavevmode\vspace{-1.5em}}
\setlist[description]{style=nextline}
% For highlighting options to programs 
% according to man-pages.
\definecolor{optcol}{rgb}{0.70, 0, 0}
\newcommand{\opt}[1]{{\color{optcol}\textit{\texttt{#1}}}}

\newcommand{\link}[1]{{\color{green}\texttt{\href{#1}{#1}}}}
\newcommand{\knng}{$k\text{-}$NNG}

\usepackage{sectsty}
\subsubsectionfont{\color{blue}}  % sets colour of subsubsections



%%%%% Shrug! Courtesy: https://tex.stackexchange.com/questions/279100/typeset-the-shrug-%C2%AF-%E3%83%84-%C2%AF-emoji

\newcommand{\shrug}[1][]{%
\begin{tikzpicture}[baseline,x=0.8\ht\strutbox,y=0.8\ht\strutbox,line width=0.125ex,#1]
\def\arm{(-2.5,0.95) to (-2,0.95) (-1.9,1) to (-1.5,0) (-1.35,0) to (-0.8,0)};
\draw \arm;
\draw[xscale=-1] \arm;
\def\headpart{(0.6,0) arc[start angle=-40, end angle=40,x radius=0.6,y radius=0.8]};
\draw \headpart;
\draw[xscale=-1] \headpart;
\def\eye{(-0.075,0.15) .. controls (0.02,0) .. (0.075,-0.15)};
\draw[shift={(-0.3,0.8)}] \eye;
\draw[shift={(0,0.85)}] \eye;
% draw mouth
\draw (-0.1,0.2) to [out=15,in=-100] (0.4,0.95); 
\end{tikzpicture}}


%mainly for emojis
\usepackage{tikzsymbols}

% for colored frame boxes
\newcommand{\cfbox}[2]{%
    \colorlet{currentcolor}{.}%
    {\color{#1}%
    \fbox{\color{currentcolor}#2}}%
}

