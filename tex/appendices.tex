\begin{appendices}
\renewcommand{\thesection}{\Roman{section}\;\;}
%%% Just continue as you would usually have while writing the the main part of the document. 
\section{Installing and running the Code}
\label{sec:install}

The program can be downloaded from Github: \url{https://github.com/gtelang/tspnng}. Alternatively
open a terminal and run the command, \texttt{git clone https://github.com/gtelang/tspnng.git}

The only other prerequisites for running the code, are the 
\href{https://www.anaconda.com/products/individual}{Anaconda} distribution of Python 3 
and a couple of other packages.  To check if the Python executable is in your path (and that it is Python 3.7\texttt{+}) 
run the command \verb|python --version|. If it succeeds, you have installed Anaconda! 

The additional packages required can be installed by: 

\begin{quote}
\color{blue}
\texttt{pip install colorama prettytable tsp} \footnote{If you don't have superuser access during installation, add the flag \texttt{\color{red} \texttt{-{}-}user} at the end}   \\
\texttt{git clone https://github.com/jvkersch/pyconcorde} \\
\texttt{cd pyconcorde}\\
\texttt{pip install -e .}
\end{quote}

%#If the installation of \texttt{pyconcorde} fails, that's okay!; the code will switch to the much slower Python based TSP solver that also computes optimal solutions. 
%This slower package can compute tours of size upto 30 to 40 in a few seconds; lareger point-clouds will take more than a couple of minutes.  

To run the program, \texttt{cd} into the code's top-level folder, then type \footnote{On Windows replace, the forward slash `/` by `\textbackslash`}
any one of: 

\begin{itemize}
\item \verb|python src/main.py --interactive|
\item \verb|python src/main.py --batchtest|
\item \verb|python src/main.py --file <points.yaml>|
\end{itemize}

\subsection{Interactive Mode}

In this mode, one can mouse-in points onto a canvas (with double-clicks), run various network algorithms 
and render the them onto a GUI canvas. 

Once you finish mousing in the points, press \verb|`i`|; that will open up a prompt at the terminal, asking which 
network do you want to compute on those points. Enter the code in the brackets. 

If you don't want to mouse-in points, and just want to plaster uniformly distributed random points on the canvas, 
press \verb|u|, and then type into the terminal the number of points. Same for non-uniform distributed points: for that
press \verb|n|. 

Please note, in \verb|interactive| mode you might see a
warning in your terminal:

\begin{quote}
\color{blue}
\verb|CoreApplication::exec: The event loop is already running|
\end{quote}

Please ignore it! It doesn't affect any of the results. Something in the
the internals of Matplotlib that uses Qt triggers that message. \shrug. 

If you have any trouble --- or detect a bug! ---  we can hash things out on Slack, Github or email.

\section{Laundry-list of Questions/Variants/Conjectures}
 \label{sec:questions}


\begin{description}
\item[\color{red} HAMILTONICITY STRUCTURE] We know that the Delaunay Triangulation of a set of points need not be Hamiltonian. In fact \textit{detecting}
      Hamiltonicity of a Delaunay Triangulation is famously $NP$-complete \cite{dillencourt1996finding} 

      Two useful facts before we proceed: 

     \begin{description}
       \item[Folklore] The cube of any connected, unweighted, graph is Hamiltonian. 
             \footnote{It is sufficient to prove this fact for any tree, and then use it on the spanning tree of the given graph}. 
       \item[Fleischner's theorem  \cite{georgakopoulos2009short}] The square of 
             any 2-vertex connected, unweighted, graph is Hamiltonian. 
          \footnote{This last theorem certainly applies to Delaunay triangulations of general point-sets. By the square of a delaunay triangulation, I mean to say: throw away the edge weights and consider the square of the underlying unweighted graph. Once the new edges are added, consider them weighted with the natural euclidean distance between their endpoints}
     \end{description}

     And so the following questions are natural:

     \begin{itemize}
      \item Based on the experiments results shown in this report, can we 
            claim  $TSP \subseteq DT^{k}$ or $TSP \subseteq MST^{k}$ in $\RR^2$ 
           for a \textit{small} constant $k$? I'd wager $k=2,3$ or, at worst, some 
           very slowly growing function of $n$ \footnote{$\log (n)$ maybe?}

      \item For $G=MST^3$, how good is the any (or the shortest??) Hamilton cycle through $G$
            in approximating the TSP? \footnote{This is a cute vertex analog of the standard edge-doubling based 2-OPT heuristic, but is detecting such a cycle for a small MST power polytime? I'd bet yes. Probably the FPT experts have something to say on this topic.}  
            Surely, this must be known, right?  For the arrangement of $n$ points at the roots of unity suggests that the 
            approximation could be as bad as 3. 

       \item What is the likelihood of $MST^2$ of $n$ points $\in \RR^2$ in general position being Hamltonian? 
             Any characterization of such point-sets? 

       \item Given a set of points in $\RR^2$, does *ANY* 
             (weakly/strongly) simple polygon and *ANY* triangulation on those 
             points have an edge in common? This is surely not true, right?!
      \end{itemize}


\item[\color{red} TSP $\rarr$ Delaunay] 

\begin{itemize}
   \item  Suppose by some magic black box (Concorde! \Winkey) one has obtained the TSP tour through $n$ points.  
          Can we then compute the Delaunay Triangulation (or MST. Recall $MST \subseteq DT$  ) of the point-set 
          in $O(n)$ time? It certainly seems so going by the high number of edges in the TSP common to the Delaunay 
          and MST as suggested by the tests in this writeup. Surely the $TSP \rarr MST$ question in $O(n)$ time has been studied?
          Maybe the Okabe book on Voronoi etc. has something on this? 

          This is kind of resonant of Melkman (i.e. simple poly chain $\rarr$ convex hull in $O(n)$). Samir Khuller also had work (SODA 1992, iirc) 
          in which he was able to compute a $O(1)$ (1.25, was it?) approximations to low-weight spanning trees of max degree 3 on point-sets in $\RR^2$, 
          from an MST in linear time. That work was based on a very nice generalization of the triangle inequality 
          \footnote{which bounded the perimeter of a triangle in terms of the sums of the distances of a point $X$ to the points of that triangle. It generalizes the $\Delta$ inequality, when we set $X$ to any of the vertices of the triangle. I forget the exact coefficients of his inequality which was linear or something}.      
          Might that be exploitable here? Check his work. 

          It's, of course, complete baloney to compute triangulations or MSTs in this manner but when did \textit{that} ever stop a computer scientist 
          from being metaphysical?

   \item (Bizarre idea!!) Is the Min Weight (MW), or even better, the Min-Angle Maximizing (MAM) Triangulation of the TSP and its external pockets close 
          to being Delaunay in some sense\footnote{e.g. how good is the maximum value of the minimum angle in a triangle over all triangles different from OPT}?  
          Note that both MW and MAM triangulations of simple polygons can be computed by D.P. Should be fun to try this out experimentally. See also Sandor's 
          ALENEX paper from 2015 where he did something about computing "bad" triangulations (iirc he wanted to minimize maximum angle i.e. make triangle skinny or sth), 
          maybe compare against the triangulations obtained from his code, just to see how bad the TSP $\rarr$ Delaunay heuristic is compared 
          to the ``bad end'' of the spectrum? 
\end{itemize}  

 %\item[Approximation by Sharing] If a simple polygon $P$ on $n$ points shares $\theta n$ edges with the Delaunay 
%      Triangulation, for some constant (i.e. independent of $n$) $\theta$ close to 1 
%      are there bounds on the approximation factor (as a function of $\theta$) of how good $P$ is compared to TSP? 


\end{description}




\end{appendices}
