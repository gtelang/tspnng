\begin{appendices}
\renewcommand{\thesection}{\Roman{section}\;\;}
%%% Just continue as you would usually have while writing the the main part of the document. 
\section{Running the Code}

Ideally, you will just need to download and install \href{https://www.anaconda.com/products/individual}{Anaconda} and a couple of other Python packages.  

To check if the Python executable is in your path and that it is Python 3\texttt{+}, run the command \verb|python --version| 
inside your terminal emulator. If it succeeds in printing something similar to ``\verb|Python 3.7.3|'' --- as it did on my laptop --- then 
you have successfully installed the Anaconda distro. 

The additional packages required can be installed executing in succession the following commands:
\begin{quote}
\color{blue}

\texttt{pip install colorama tsp} \footnote{If you don't have superuser access during installation, add the flag \texttt{\color{red} \texttt{-{}-}user}}   \\
\texttt{git clone https://github.com/jvkersch/pyconcorde} \\
\texttt{cd pyconcorde}\\
\texttt{pip install -e .}
\end{quote}

If the installation of pyconcorde fails, that's okay, the code will switch to the much slower Python based TSP solver. It can compute tours
of size upto 30, 40 fairly quickly, after which the computation might takes more than a couple of minutes.  

To run the code, \texttt{cd} into the code's top-level folder, then type \footnote{On Windows replace, the forward slash `/` in the commands below by `\textbackslash`}
any one of the following commands into your terminal. 

\begin{itemize}
\item \verb|python src/main.py --interactive|
\item \verb|python src/main.py --batchtest|
\item \verb|python src/main.py --file <points.yaml>|
\end{itemize}

\subsection{Interactive Mode}

In this mode, one can mouse-in points onto a canvas (with double-clicks) and, various network algorithms and render the 
resulting network. 

Once you finish mousing in the points, press \verb|`i`|, and that will open up a prompt at the terminal, asking which 
network do you want to compute on those points. Enter the code in the brackets. 

If you don't want to mouse-in points, and just want to plaster a few uniformly distributed random points on the canvas, 
press \verb|`u`|, and then type into the terminal the number of points. (Same for non-uniform distributed points, but for that
you must press \verb|`n`|)

Please note, while  running the  code in `\verb|interactive|` mode you will a
warning message:

\begin{quote}
\color{blue}
\verb|CoreApplication::exec: The event loop is already running|
\end{quote}

Please ignore it! It doesn't affect any of the results. Something in the
the internals of the Matplotlib that uses the Qt library triggers that warning. I 
am not sure what. The message doesn't pop up
when I use Python 2.7 \footnote{Note that Python 2 has been deprecated in favour of Python 3\texttt{+}, since January 2020, } 
instead of Python 3.7. 

If you have any trouble --- or detect a bug! ---  we can hash things out on Slack, Github or email.

\section{Laundry-list of Questions/Variants/Conjectures}
\end{appendices}
