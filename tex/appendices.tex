\begin{appendices}
\renewcommand{\thesection}{\Roman{section}\;\;}
%%% Just continue as you would usually have while writing the the main part of the document. 
\section{Installing and running the Code}
\label{sec:install}

The program can be downloaded from Github: \url{https://github.com/gtelang/tspnng}. Alternatively
open a terminal and run the command, \texttt{git clone https://github.com/gtelang/tspnng.git}

The only other prerequisites for running the code, are the 
\href{https://www.anaconda.com/products/individual}{Anaconda} distribution of Python 3 
and a couple of other packages.  To check if the Python executable is in your path (and that it is Python 3.7\texttt{+}) 
run the command \verb|python --version|. If it succeeds, you have installed Anaconda! 

The additional packages required can be installed by: 

\begin{quote}
\color{blue}
\texttt{pip install colorama prettytable tsp} \footnote{If you don't have superuser access during installation, add the flag \texttt{\color{red} \texttt{-{}-}user} at the end}   \\
\texttt{git clone https://github.com/jvkersch/pyconcorde} \\
\texttt{cd pyconcorde}\\
\texttt{pip install -e .}
\end{quote}

%#If the installation of \texttt{pyconcorde} fails, that's okay!; the code will switch to the much slower Python based TSP solver that also computes optimal solutions. 
%This slower package can compute tours of size upto 30 to 40 in a few seconds; lareger point-clouds will take more than a couple of minutes.  

To run the program, \texttt{cd} into the code's top-level folder, then type \footnote{On Windows replace, the forward slash `/` by `\textbackslash`}
any one of: 

\begin{itemize}
\item \verb|python src/main.py --interactive|
\item \verb|python src/main.py --batchtest|
\item \verb|python src/main.py --file <points.yaml>|
\end{itemize}

\subsection{Interactive Mode}

In this mode, one can mouse-in points onto a canvas (with double-clicks), run various network algorithms 
and render the them onto a GUI canvas. 

Once you finish mousing in the points, press \verb|`i`|; that will open up a prompt at the terminal, asking which 
network do you want to compute on those points. Enter the code in the brackets. 

If you don't want to mouse-in points, and just want to plaster uniformly distributed random points on the canvas, 
press \verb|`u`|, and then type into the terminal the number of points. Same for non-uniform distributed points: for that
press \verb|`n`|. 

Please note, in `\verb|interactive|` mode you might see a
warning in your terminal:

\begin{quote}
\color{blue}
\verb|CoreApplication::exec: The event loop is already running|
\end{quote}

Please ignore it! It doesn't affect any of the results. Something in the
the internals of Matplotlib that uses Qt triggers that message. \shrug. 

If you have any trouble --- or detect a bug! ---  we can hash things out on Slack, Github or email.

\section{Laundry-list of Questions/Variants/Conjectures}
 \label{sec:questions}

\begin{itemize}
\item We know that the Delaunay Triangulation of a set of points need not be Hamiltonian. In fact \textit{detecting}
      Hamiltonicity of a Delaunay Triangulation is famously $NP$-complete \cite{dillencourt1996finding} 

      Two natural questions suggest themselves:

     \begin{itemize}
       \item The cube of any connected (unweighted) graph is known to be Hamiltonian. \footnote{It is sufficient to prove this fact for any tree, and then use it on the spanning tree of the given graph}. 
             Is the \textit{square} of the underlying unweighted graph of the DT (or *any* triangulation of points, for that matter) Hamiltonian? 
             Surely this should be simple to show?! There are some results on testing if squares of graphs can be Hamilonian, (I remember Estie mentioning them in passing once) but 
             I can't recall them at the moment. Maybe we can exploit \cite{dillencourt1996finding} to find counter-examples?
        \item More generally, is $TSP \subseteq DT^{k}$ for a small fixed $k$? I'd wager $k=2,3$ or some slowly growing function of $n$ going by some of 
             the experimental observations in this report,. \footnote{Even if true, this observation can only be useful in $\RR^2$; it is worthless in higher dimensions because of the combinatorial complexity of delaunay arrangements}
     \end{itemize}

 
\end{itemize}

\end{appendices}
