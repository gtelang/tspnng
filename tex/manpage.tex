\vspace{5mm}
\begin{description}
  \item[SYNOPSIS] Does the Euclidean TSP for a finite set of points $P$ share an edge with $P$'s nearest neighbor graph? \footnote{In this article, we will assume the NNG to be undirected i.e. after constructing the nearest neighbor graph for a point-set we will throw away the edge directions.}
     Or its $k$-NNG? Or the Delaunay Graph? 
     Or indeed any poly-time computable graph spanning the input points? We investigate
     this question experimentally by checking the validity of this conjecture for  
     various instances  in TSPLIB, for which the optimal solutions
     have been provided and for other synthetic data-sets (e.g. uniformly and non-uniformly generated points)
     for which we can compute optimal or near-optimal tours using Concorde. 
                   
  \item[DESCRIPTION] This question suggested itself to the author while working on the Horsefly problem, 
     a generalization of the famously $NP$-hard Travelling Salesman Problem \footnote{In this report by ``$TSP$'', we mean $TSP$-cycle and not $TSP$-path, although the question is still interesting for the path case. One reason for focusing only on the path case, is that the TSPLIB bank only mentions optimal cycle solutions and not optimal path solutions, which can be structurally quite different! Also Concorde, the main library used to generate any TSP solutions also outputs cycles.}. 
     One line of attack was to get at some kind of structure theorem by identifying  a candidate set of good edges from which a near-optimal solution to the 
     horsefly problem could be constructed. But first off, would this approach work for the special case of the  TSP?  Answering
     \textit{``$TSP \cap NNG \stackrel{?}{=} \varnothing$''} seemed like a good place to start.  However, all attempts  
     at constructing examples in which the intersection is \textit{empty} failed . And so did a
     literature search. The closest matching reference we found was \cite{hougardy2014edge} which \textit{eliminates} edges 
     that cannot be part of a Euclidean TSP tour on a given instance of points, based on checking a few simple, local geometric inequalities. 

     Bill Cook, the author of Concorde\cite{applegate2009certification},  on hearing about this problem from Prof. Mitchell said that, if true, it could be used 
     to speed up some of the existing experimental TSP heuristics.  
     \footnote{Note that the landmark PTAS'es for the TSP, such as those of Mitchell \cite{mitchell1999guillotine} and Arora\cite{arora1996polynomial},  are too complicated to be put into code (yes, even Python!). On the other hand, the Concorde library uses a whole kitchen-sink of practical techniques such as $k$-local swaps, branch-and-bound, branch-and-cut to generate  near-optimal (if not optimal) tours relatively quickly. However,it would be interesting to investigate the behavior of the various graphs with respect to the techniques used in the PTAS'es of Mitchell and Arora. Maybe we can augment them with the probabilistic method to prove the existence of an intersection??}

     To spur our intuition,  we investigate the conjecture experimentally in this short report 
     \footnote{This report has been written as a literate program \cite{knuth1984literate,ramsey2008noweb} to weave together the code, explanations and generated data into the same document. Feedback on the author's preliminary stab at literate programming is most welcome!}
     using TSPLIB and Concorde in tandem. TSPLIB is an online collection of medium to large size instances for the Euclidean, Metric and other several variants of the TSP 
     for which optimal solutions have been obtained using powerful heuristics implemented in libraries like Concorde or Keld-Helsgaun;
     the certificate of optimality for these instances (as always!) comes from comparing the tour-length of the computed against 
     a lower bound computed by those very heuristics. 

     For starters,  we investigate the following questions \footnote{Experimental answers to other questions will be barnacled onto the report as it grows}: 
     for each symmetric 2-D Euclidean TSP instance from TSPLIB for which we have an optimal solution, does

     \begin{itemize}
     \item $TSP \cap (k{\text -})NNG \stackrel{?}{=} \varnothing$, for $k=1,2,\ldots$
     \item $TSP \cap \textit{Delaunay Graph} \stackrel{?}{=} \varnothing$
     \item For question 1, in the cases that the intersection is non-empty, what fraction (a fourth?, a fifth?)of the $n$ 
           edges of a TSP-tour share its edges with the  $k$-NNG does the TSP intersect for various values of $k$? 
     \item Are there any structural patterns observed in the intersections? Specifically, does \textit{at least } 
           one edge from the intersection have a \textit{vertex} incident to the convex hull? \footnote{This indeed seemed to be the case in all the author's failed attempts at a counter-example, and so we are looking for a proof/disproof for this special case of the conjecture}
           More generally, is this true for every layer of the onion?
     \end{itemize}

     See also Appendix A for a running wishlist of questions that come out during discussions. 

     As an aid in constructing possible counter-examples, a GUI interface is provided to mouse-in points and then 
     run the Concorde heuristic on it. 
     
     The Python 3.7\texttt{+} code used to generate the data and figures in this paper has been attached to this pdf. If you don't have a Python distribution 
     please download the freely available  \href{https://www.anaconda.com/products/individual}{Anaconda}  distro; it comes 
     with most of the batteries included. You will also need to install a couple of other packages. See Appendix I. 

     \textit{Yalla}, what are we waiting for?! Let’s go! 
\end{description}
