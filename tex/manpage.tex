\vspace{5mm}
\begin{description}
  \item[SYNOPSIS] Does the Euclidean TSP for a finite set of points $P$ share an edge with $P$'s nearest neighbor graph? \footnote{In this article, we will assume the NNG to be undirected i.e. after constructing the nearest neighbor graph for a point-set we will throw away the directions of the edges.}
     Or its $k$-NNG? Or the Delaunay Graph? 
     Or indeed any poly-time computable graph spanning the input points? We investigate
     this question experimentally by checking the validity of this conjecture for  
     various instances  in TSPLIB, for which the optimal solutions
     have been provided. 
                   
  \item[DESCRIPTION] This question suggested itself to the author while working on the Horsefly problem, itself is 
     a generalization of the famously $NP$-hard Travelling Salesman Problem \footnote{In this report by ``$TSP$'', we mean $TSP$-cycle and not $TSP$-path, although the question is still interesting for the path case. One reason for focusing only on the path case, is that the TSPLIB bank only mentions optimal cycle solutions and not optimal path solutions, which can be structurally quite different! Also Concorde, the main library used to generate any TSP solutions also outputs cycles.}. 
     One line of attack was to get at some kind of ``structure theorem'' by identifying  a candidate set of ``good'' edges from which a near-optimal solution to the 
     horsefly problem could be constructed. But first off, would this approach work for the special case of the  TSP?  Answering
     \textit{``$TSP \cap NNG \stackrel{?}{=} \varnothing$''} seemed like a good place to start.  However, all attempts  
     at constructing counter-examples in which the intersection is \textit{empty} have, thus far, failed. And so has a cursory 
     literature search. Bill Cook (the author of Concorde) on hearing about this problem from Prof. Mitchell said that, if true,
     it could be used to speed up some of the existing TSP heuristics. 

     To spur our intuition,  we investigate the conjecture experimentally in this short report 
     \footnote{This report has been written as a literate program to weave together the code, explanations and generated data into the same document. Feedback on the author's preliminary stab at literate programming is most welcome!}
     using TSPLIB and Concorde in tandem. TSPLIB is an online collection of medium to large size instances for the Euclidean, Metric and other several variants of the TSP 
     for which optimal solutions have been obtained using powerful heuristics implemented in libraries like Concorde or Keld-Helsgaun;
     the certificate of optimality for these instances (as always!) comes from comparing the tour-length of the computed against 
     a lower bound computed by those very heuristics. 

\newpage

     For starters,  we investigate the following questions \footnote{Experimental answers to other questions will be barnacled to the report as it keeps (hopefully!) growing}: 
     for each symmetric 2-D Euclidean TSP instance from TSPLIB for which we have an optimal solution, does

     \begin{itemize}
     \item $TSP \cap (k{\text -})NNG \stackrel{?}{=} \varnothing$, for $k=1,2,\ldots$
     \item $TSP \cap \textit{Delaunay Graph} \stackrel{?}{=} \varnothing$
     \item For question 1, in the cases that the intersection is non-empty, what fraction (a fourth?, a third?)of the $n$ 
           edges of a TSP-tour share its edges with the  $k$-NNG does the TSP intersect for various values of $k$? 
     \item Are there any structural patterns observed in the intersections? Specifically, does \textit{at least } 
           one edge of the nearest neighbor graph have an edge with a \textit{vertex} incident to the convex hull? \footnote{This indeed seemed to be the case in all the author's failed attempts at a counter-example, and so are looking for a proof/disproof for this special vase of the conjecture}
           More generally, is this true for every layer of the ``onion''?
     \end{itemize}

     See also Appendix A for a running wishlist of questions that come out during discussions. 

     As an aid in constructing possible counter-examples, a GUI interface is provided to mouse-in points and then 
     run the Concorde heuristic on it. 
     
     The Python 3.7+ code used to generate the data and figures in this paper has been attached to this pdf. If you don't have a Python distribution 
     please download the freely available  \href{https://www.anaconda.com/products/individual}{Anaconda}  distro, that comes 
     with all the ``batteries included''.

     Instructions for running the code have been relegated to the appendix. All development and testing was done on  a Linux machine;
     minimal modification (if at all!) would be needed to run it on  Windows or Mac. In any event, the boring technical issues can 
     be hashed out on Slack. 
     
     \textit{Yalla}, let’s go! 
\end{description}
